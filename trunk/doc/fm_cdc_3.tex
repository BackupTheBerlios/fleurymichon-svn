Comme indique precedemment, le projet se divise en deux parties principales.

\section{Le serveur IRC}

L'application serveur IRC constitue l'application principale du projet. Il s'agit d'un \emph{daemon}, c'est-a-dire d'une application qui reside en arriere-plan sur la machine sur laquelle elle est executee et qui est responsable de la gestion d'une ou plusieurs taches specifiques, la gestion de communications par le biais du reseau dans notre cas. C'est un service systeme qui pourra etre configure pour un lancement automatique au demarrage de la machine. La version Windows s'integrera dans le systeme de gestion des services de Windows. De plus, le systeme de gestion de modules assure la flexibilite de l'application, ce qui permet a l'administrateur de l'adapter aux besoins de ses utilisateurs. Il s'agit donc d'une application qui se destine aux administrateurs de serveurs, mais aussi plus largement a une personne desireuse de mettre en place un reseau de discussion.

\section{Le client IRC}

Bla bla bla...

\section{R�partition des t�ches}

\subsection{T�ches non assign�es}

\begin{itemize}
\item[*] Site web
\item[*] Repository pour partager le code source
\item[*] Serveur
\item[--] Daemon
\item[--] Gestion r�seau
\item[--] Modules
\item[---] Bots de gestion
\item[---] Pont NetSoul
\item[*] Client
\item[--] Interface
\item[--] Gestion r�seau
\item[*] Packaging
\end{itemize}

\subsection{Premi�re soutenance}

Bla bla bla...

\subsection{Deuxi�me soutenance}

Bla bla bla...

\subsection{Troisi�me soutenance}

Bla bla bla...

\subsection{Soutenance finale}

Bla bla bla...
