Comme indiqu� pr�cedemment, le projet se divise en deux parties distinctes : le serveur et le client.

\section{Le serveur IRC}

L'application serveur IRC constitue l'application principale du projet. Il s'agit d'un \emph{daemon}, c'est-�-dire d'une application qui r�side en arriere-plan dans la m�moire de la machine sur laquelle elle est �x�cut�e et qui est responsable de la gestion d'une ou de plusieurs t�ches sp�cifiques, la gestion de communications par le biais du r�seau dans notre cas. C'est un service syst�me qui pourra �tre configur� pour un lancement automatique au d�marrage de la machine. La version Windows s'int�grera dans le syst�me de gestion des services de Windows. De plus, le syst�me de gestion des modules assure la flexibilit� de l'application, ce qui permet � l'administrateur de l'adapter aux besoins de ses utilisateurs. Il s'agit donc d'une application qui se destine aux administrateurs de serveurs, mais aussi plus largement � toute personne desireuse de mettre en place un r�seau de discussion en temps r�el.

\section{Le client IRC}

Le client IRC disposera d'une interface graphique ergonomique et fonctionnelle. Il permettra ainsi d'acc\'{e}der en quelques clics \`{a} toutes les fonctinnalit\'{e}s du serveur. Ainsi, il satisfera autant l'utilisateur novice que l'utilisateur exp\'{e}riment\'{e}, car il disposera d'options de configuration avanc\'{e}es.

\newpage

\section{R�partition des t�ches}

\subsection{Premi�re soutenance}

Site Web \\
Repository pour partager le code source \\

\begin{center}
\begin{tabular}{||l||l||}
\hline
Sergue� & \multirow{2}{*}{Impl�mentation basique de l'application serveur} \\
\cline{1-1}
Benoit & \\
\hline
Lionel & \multirow{2}{*}{Impl�mentation basique de l'application client} \\
\cline{1-1}
Nicolas & \\
\hline
\end{tabular}
\end{center}

\subsection{Deuxi�me soutenance}

\begin{center}
\begin{tabular}{||l||l||}
\hline
Sergue� & \multirow{2}{*}{Version op�rationnelle de l'application serveur} \\
\cline{1-1}
Benoit & \\
\hline
Lionel & \multirow{2}{*}{Version op�rationnelle de l'application client} \\
\cline{1-1}
Nicolas & \\
\hline
\end{tabular}
\end{center}

\subsection{Troisi�me soutenance}

Modules (bots de gestion, pont NetSoul) \\

\begin{center}
\begin{tabular}{||l||l||}
\hline
Sergue� & Bla \\
\hline
Benoit & Bla \\
\hline
Lionel & Bla \\
\hline
Nicolas & Bla \\
\hline
\end{tabular}
\end{center}

\subsection{Soutenance finale}

Portage \\
Packaging \\

\begin{center}
\begin{tabular}{||l||l||}
\hline
Sergue� & Bla \\
\hline
Benoit & Bla \\
\hline
Lionel & Bla \\
\hline
Nicolas & Bla \\
\hline
\end{tabular}
\end{center}
