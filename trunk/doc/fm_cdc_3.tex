Comme indique precedemment, le projet se divise en deux parties principales.

\section{Le serveur IRC}

L'application serveur IRC constitue l'application principale du projet. Il s'agit d'un \emph{daemon}, c'est-�-dire d'une application qui r�side en arriere-plan dans la m�moire de la machine sur laquelle elle est �x�cut�e et qui est responsable de la gestion d'une ou de plusieurs t�ches sp�cifiques, la gestion de communications par le biais du r�seau dans notre cas. C'est un service syst�me qui pourra �tre configur� pour un lancement automatique au d�marrage de la machine. La version Windows s'int�grera dans le syst�me de gestion des services de Windows. De plus, le syst�me de gestion des modules assure la flexibilit� de l'application, ce qui permet � l'administrateur de l'adapter aux besoins de ses utilisateurs. Il s'agit donc d'une application qui se destine aux administrateurs de serveurs, mais aussi plus largement � toute personne desireuse de mettre en place un r�seau de discussion en temps r�el.

\section{Le client IRC}

Bla bla bla...

\section{R�partition des t�ches}

\subsection{T�ches non assign�es}

\begin{itemize}
\item[*] Site web
\item[*] Repository pour partager le code source
\item[*] Serveur
\item[--] Daemon
\item[--] Gestion r�seau
\item[--] Modules
\item[---] Bots de gestion
\item[---] Pont NetSoul
\item[*] Client
\item[--] Interface
\item[--] Gestion r�seau
\item[*] Packaging
\end{itemize}

\subsection{Premi�re soutenance}

Bla bla bla...

\subsection{Deuxi�me soutenance}

Bla bla bla...

\subsection{Troisi�me soutenance}

Bla bla bla...

\subsection{Soutenance finale}

Bla bla bla...
