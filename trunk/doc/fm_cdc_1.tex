\section{Origine du projet}

Ce soir la, la nuit paraissait plus noire que d'habitude et les vents froids d'octobre faisaient claquer violemment les volets en bois de ma chambre. Le moment semblait ideal pour rejoindre mes amis epiteens, afin de partager avec eux les joies du tarot a quatre on-line. Quelques minutes de jeu ont suffit a rendre ce qui devait etre un moment d epanouissement personnel en recreation pour primaire en crise. En effet, la vitesse a laquelle apparaissait les insultes sur mon ecran s'approchait d'une compilation moderne. 
\\
Ainsi, nous eumes l'idee de creer notre propre ring, un monde meilleur pour le flood et autres paroles parasites. FleuryMichon etait ne!    

\section{Description du projet}

Bla bla bla...

\section{Composition du groupe}

\noindent Sergue� Milechine (Info-Sp� A1) \\
\verb+<milech_s@epita.fr>+ \\
\begin{footnotesize}
Projet Sup : \bsc{Bar Manager}, \emph{jeu de gestion de bar}
\end{footnotesize} \\ \\
Benoit Menet (Info-Sp� A1) \\
\verb+<menet_b@epita.fr>+ \\
\begin{footnotesize}
Projet Sup : \bsc{Mad Sheep}, \emph{/* ******** */}
\end{footnotesize} \\ \\
Lionel Herbin (Info-Sp� A1) \\
\verb+<herbin_l@epita.fr>+ \\
\begin{footnotesize}
Projet Sup : \bsc{BlaiRWitcH}, \emph{/* ******** */}
\end{footnotesize} \\ \\
Nicolas Vernot (Info-Sp� B1) \\
\verb+<vernot_n@epita.fr>+ \\
\begin{footnotesize}
Projet Sup : \bsc{Chu Chu Rocket}, \emph{casse-tete/puzzle}
\end{footnotesize}
