\section{Vers un serveur fonctionnel}

\section{Corrections effectu�es}

\section{Impl�mentation de nouvelles commandes}

\subsection{La commande \emph{PRIVMSG}}

La commande \emph{PRIVMSG} est utilis�e lorsque un utilisateur d�sire s'adresser � un ou plusieurs utilisateurs du r�seaux, mais �galement � un ou plusieurs canaux.

\subsection{La commande \emph{LIST}}

La commande \emph{LIST} permet d'obtenir la liste d'un ou plusieurs canaux avec leur \emph{topic} (sujet), except� si ce dernier est priv�. 

\subsection{La commande \emph{KICK}}

La commande \emph{KICK} permet � un op�rateur de sortir un utilisateur du canal.

\subsection{La commande \emph{NAMES}}

La commande \emph{NAMES} est utilis�e pour obtenir la liste des utilisateurs pr�sents sur un canal.

\subsection{La commande \emph{WHO}}

\subsection{La commande \emph{TOPIC}}

La commande \emph{TOPIC} permet � un utilisateur de d�finir ou de voir le sujet d'un canal. 

\subsection{La commande \emph{MODE}}
La commande \emph{MODE} permet de changer le statut d'un utilisateur (op�rateur, invisible, \ldots) ou d'un canal (priv�, nombre de personnes maximum, \ldots).

\subsubsection{Modes des canaux}
\subsubsection{Modes des utilisateurs}

\section{Gestion des erreurs}

\section{Fichier de configuration}

Dans le but de fournir une application graphique de param�trage du serveur, nous avons mis en place un fichier d'initialisation 
des longueurs maximum des chaines de caract�res telles que le nom du serveur, le \emph{nick} de l'utilisateur\ldots, ainsi que 
des valeurs num�riques comme le port d'�coute.

