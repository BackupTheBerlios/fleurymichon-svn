\section{Moyens employ�s}

\subsection{Plate-forme}

L'application serveur, ainsi que l'application client munie d'une interface graphique, seront developp�es sous et optimis�es pour Linux. Nous allons nous efforcer de garder le code aussi portable que possible afin de pouvoir fournir une version Windows du projet pour la soutenance finale dans la mesure du r�alisable.

\subsection{Langage}

L'application serveur sera impl�ment�e en langage C. Le compilateur employ� sera \emph{gcc}, le c�l�bre compilateur C du projet GNU. \\
L'application client nec�ssitant une interface graphique ergonomique, nous avons d�cid� de l'impl�menter en langage C++. Pour ce faire, nous ferons appel a une biblioth�que de classes et de fonctions adapt�e, telle que Qt ou wxWidgets.

\subsection{Application orient�e r�seau}

Puisque notre projet est totalement orient� vers la communication en r�seau, nous travaillerons avec des biblioth�ques standardis�es. Nous devrons donc utiliser principalement les fonctions de traitement de connexions (\emph{sys/socket.h}) de la biblioth�que GNU, fournie avec presque toutes les distributions Linux. \\
La version Windows devra utiliser la biblioth�que \emph{Winsock} pour g�rer les connexions r�seau.

\subsection{Application �volutive}

L'application serveur disposera d'un syst�me flexible de gestion de modules. Il s'agit de biblioth�ques qui s'integrent dans l'application et qui permettent d'�largir le champ des fonctionnalit�s disponibles. Nous en avons envisag� plusieurs qui seront developp�s dans le cadre du projet.

\subsubsection{Robots de gestion}

Ce module �mulera sur le serveur deux robots d�di�s � la gestion des comptes utilisateurs. Ainsi, un utilisateur qui se connecte au serveur peut cr�er un compte identifi� par son surnom et ainsi prot�ger son utilisation. De plus, le serveur m�morisera les permissions de l'utilisateur sur un canal de discussion dont il ferait partie des membres privilegi�s.

\subsubsection{Pont NetSoul}

Bla bla bla...

\newpage

\section{But et int�r�t du projet}

\subsection{Sergue� \bsc{Milechine}}

Je trouve int�ressant de d�velopper une application de communication en r�seau comme projet parce que les applications les plus utilisees de nos jours doivent s'orienter vers l'exterieur. Ainsi, cela nous permet de nous initier � la programmation r�seau, indispensable pour r�aliser des applications op�rationnelles depuis la d�mocratisation de l'acc�s Internet.


\subsection{Benoit \bsc{Menet}}

Ce projet me tient a coeur dans le sens ou il met clairement en avant la programmation reseau. Etant un novice en la matiere, je me confronte donc a un double defi : celle de m'informer sur un sujet encore obscur aujourd'hui, et celle de realiser un projet sur le moyen terme, mettant en avant certaines competences acquises au cours de mon annee de Sup. Un serveur/client IRC me semble etre un projet adapte au niveau general du groupe, ce qui permettra, je l'espere, de travailler efficacement ensemble.            


\subsection{Lionel \bsc{Herbin}}

Le fait de faire un projet sur irc permet de s'orienter dans la branche communication d'internet. En faisant cela, nous devrons donc nous lancer dans la programmation reseau ainsi que l'apprentissage du fonctionnement d'irc, or j'ai peu de connaissances dans ces deux domaines malgr� que cela m'int�resse et que j'ai pour habitude de m'en servir. \\
De plus, je souhaite que ce projet me permette d'avoir une bonne experience de travail de groupe que je n'ai pas forcement eu mon annee de sup.

\subsection{Nicolas \bsc{Vernot}}

Ce projet est interessant car il me permettra de d�couvrir la programmaton r�seau, et de d�velopper un projet en groupe avec toutes les contraintes que cela implique. De plus comme le projet devra etre portable, et comme le c est un language permettant de produire facilement un code illisible, je devrais apprendre a respecter certaines normes. Ce projet me permettra donc d'aqu�rir des connaissances et techniques indispensables pour le futur.
