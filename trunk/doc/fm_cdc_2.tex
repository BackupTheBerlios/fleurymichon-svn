\section{Moyens employ�s}

\subsection{Plate-forme}

L'application serveur, ainsi que l'application client munie d'une interface graphique, seront developpees sous et optimisees pour Linux. Nous allons nous efforcer de garder le code aussi portable que possible afin de pouvoir fournir une version Windows du projet pour la soutenance finale dans la mesure du realisable.

\subsection{Langage}

L'application serveur sera implementee en langage C. Le compilateur employe sera \emph{gcc}, le celebre compilateur C du projet GNU. \\
L'application client necessitant une interface graphique ergonomique, nous avons decide de l'implementer en langage C++. Pour ce faire, nous ferons appel a une bibliotheque de classes et de fonctions adaptee, telle que Qt ou wxWidgets.

\subsection{Application orient�e r�seau}

Puisque notre projet est totalement oriente vers la communication en reseau, nous travaillerons avec des bibliotheques standardisees. Nous devrons donc utiliser principalement les fonctions de traitement de connexions (\emph{sys/socket.h}) de la bibliotheque GNU, fournie avec presque toutes les distributions Linux. \\
La version Windows devra utiliser la bibliotheque \emph{Winsock} pour gerer les connexions reseau.

\subsection{Application �volutive}

L'application serveur disposera d'un systeme flexible de gestion de modules. Il s'agit de bibliotheques qui s'integrent dans l'application et qui permettent d'elargir le champ des fonctionnalites disponibles. Nous en avons envisage plusieurs qui seront developpes dans le cadre du projet.

\subsubsection{Robots de gestion}

Ce module emulera sur le serveur deux robots dedies a la gestion des comptes utilisateurs. Ainsi, un utilisateur qui se connecte au serveur peut creer un compte identifie par son surnom et ainsi proteger son utilisation. De plus, le serveur memorisera les permissions de l'utilisateur sur un canal de discussion dont il ferait partie des membres privilegies.

\subsubsection{Pont NetSoul}

Bla bla bla...

\newpage

\section{But et int�r�t du projet}

\subsection{Sergue� \bsc{Milechine}}

Je trouve int�ressant de d�velopper une application de communication en r�seau comme projet parce que les applications les plus utilisees de nos jours doivent s'orienter vers l'exterieur. Ainsi, cela nous permet de nous initier � la programmation r�seau, indispensable pour r�aliser des applications op�rationnelles depuis la d�mocratisation de l'acc�s Internet. De plus, puisque notre application serveur est concue dans une optique d'evolutivite, grace a un systeme de modules qui permettra a l'utilisateur final du projet de le personnaliser selon ses besoins, cela nous oblige a structurer rigoureusement le fonctionnement de notre projet et a communiquer raisonnablement au sein du groupe pour assurer une implementation efficace, coherente et fonctionnelle.


\subsection{Benoit \bsc{Menet}}

Ce projet me tient � coeur dans le sens ou il met clairement en avant la programmation r�seau. Etant un novice en la mati�re, je me confronte donc � un double d�fi : celle de m'informer sur un sujet encore obscur aujourd'hui, et celle de r�aliser un projet sur le moyen terme, mettant en avant certaines comp�tences acquises au cours de mon ann�e de Sup. Un serveur/client IRC me semble �tre un projet adapt� au niveau g�n�ral du groupe, ce qui permettra, je l'esp�re, de travailler efficacement ensemble.            


\subsection{Lionel \bsc{Herbin}}

Le fait de faire un projet sur irc permet de s'orienter dans la branche communication d'internet. En faisant cela, nous devrons donc nous lancer dans la programmation r�seau ainsi que l'apprentissage du fonctionnement d'irc, or j'ai peu de connaissances dans ces deux domaines malgr� que cela m'int�resse et que j'ai pour habitude de m'en servir. \\
De plus, je souhaite que ce projet me permette d'avoir une bonne exp�rience de travail de groupe que je n'ai pas forcement eu mon ann�e de sup.

\subsection{Nicolas \bsc{Vernot}}

Ce projet est interessant car il me permettra de decouvrir la programmaton reseau, et de develloper un projet en groupe avec toutes les contraintes que cela implique. De plus comme le projet devra etre portable, et comme le c est un language permettant de produire facilement un code illisible, je devrais apprendre a respecter certaines normes. Ce projet me permettra donc d'aquerir des connaissances et techniques indispensables pour le futur. 
