\section{Site Web}

Notre projet est h�berg� chez BerliOS, il s'agit d'un fournisseur de services pour d�veloppeurs qui nous apporte d'innombrables outils pour travailler sur notre projet. Nous disposons d'une interface assez compl�te permettant d'organiser l'�volution du projet, nous pouvons recevoir des demandes d'ajout de fonctionnalit�s, des demandes de support, �mettre des sondages, rapporter les bugs, discuter, publier les releases du projet, des screenshots, de la documentation \ldots

\section{Repository}

Par ailleurs, BerliOS nous offre �galement un repository Subversion. Il s'agit d'un gestionnaire de code source, qui nous permet de synchroniser notre travail � la fois rapidement et efficacement.
