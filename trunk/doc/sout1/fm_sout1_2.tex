

\section{Impl�mentation du daemon}

Pour que l'application serveur puisse fonctionner convenablement, elle doit gagner son ind�pendance par rapport � la console depuis laquelle elle est appel�e, devenant ainsi un daemon, c'est-�-dire une application r�sidant en m�moire et qui s'�x�cute invisiblement en t�che de fond. Pour ce faire, nous avons utilis� principalement les fonctions \textit{fork()} et \textit{setsid()} afin d'isoler le processus du serveur Fleury de la console depuis laquelle il a �t� �x�cut�. Ainsi, l'instance de Fleury nouvellement cr��e est d�tach�e de la console, ce qui permet � l'utilisateur d'en reprendre le contr�le pendant que le serveur tourne en background. Donc le processus de l'application serveur se duplique et le processus originel quitte, ce qui donne un nouveau processus sans p�re, qui ne risque plus d'influencer le shell, qui en �tait le processus p�re auparavant.

\section{Travaux sur les sockets}
\section{Concept de threads}

Pour plus de souplesse dans l'application serveur, nous avons fait appel au multi-threading. Un thread est un processus l�ger, ou encore une t�che qui a un r�le bien pr�cis. L'avantage des threads est qu'on peut en �x�cuter plusieurs qui fonctionneront en parall�le et qui auront acc�s aux m�mes donn�es. Ainsi, une fois le serveur initialis�, l'application comporte un thread d'�coute, qui attend les connexions des clients et qui est charg� de g�n�rer un nouveau thread de traitement pour chaque client qui se connecte au serveur. Nous avons employ� la biblioth�que de threads POSIX pour l'impl�mentation de ce syt�me de traitement, ainsi le serveur peut fonctionner dans des environnements Linux et Cygwin mais Windows n'est pas exclu pour autant puisqu'un portage de cette biblioth�que pour ce syst�me est disponible.

\section{Messages serveur}
\subsection{�tablissement de la connexion}
\subsection{Op�rations sur les canaux}





