\section{Origine du projet}

Ce soir l�, la nuit paraissait plus noire que d'habitude et les vents froids d'octobre faisaient claquer violemment les volets en bois de ma chambre. Le moment semblait id�al pour rejoindre mes amis �pit�ens, afin de partager avec eux les joies du tarot � quatre on-line. Quelques minutes de jeu ont suffit � rendre ce qui devait �tre un moment d'�panouissement personnel en r�cr�ation pour primaires en crise. En effet, la vitesse � laquelle apparaissait les insultes sur mon �cran s'approchait d'une compilation moderne. 
\\
Ainsi, nous e�mes l'id�e de cr�er notre propre ring, un monde meilleur pour le flood et autres paroles parasites. FleuryMichon �tait n�! (Ne nous demandez pas pourquoi ce nom \ldots)   

\section{Description du projet}

FleuryMichon fournira � son utilisateur un serveur (Fleury) et un client IRC (Michon). Le projet sera d�velopp� en C/C++ et utilisable sous Linux et Windows.  

\section{Composition du groupe}

\noindent Sergue� Milechine (Info-Sp� A1) \\
\verb+<milech_s@epita.fr>+ \\
\begin{footnotesize}
Projet Sup : \bsc{Bar Manager}, \emph{jeu de gestion de bar}
\end{footnotesize} \\ \\
Benoit Menet (Info-Sp� A1) \\
\verb+<menet_b@epita.fr>+ \\
\begin{footnotesize}
Projet Sup : \bsc{Mad Sheep}, \emph{jeu d'action avec des moutons}
\end{footnotesize} \\ \\
Lionel Herbin (Info-Sp� A1) \\
\verb+<herbin_l@epita.fr>+ \\
\begin{footnotesize}
Projet Sup : \bsc{BlaiRWitcH}, \emph{jeu de karting}
\end{footnotesize} \\ \\
Nicolas Vernot (Info-Sp� B1) \\
\verb+<vernot_n@epita.fr>+ \\
\begin{footnotesize}
Projet Sup : \bsc{Projet CACAGOVE}, \emph{casse-t�te/puzzle}
\end{footnotesize}

\newpage

\section{R�partition des t�ches (rappel)}

\subsection{Premi�re soutenance}

Nous mettrons en place un repository pour le partage des sources (Subversion) ainsi qu'un site Web simple et pratique.

\begin{center}
\begin{tabular}{||l||l||}
\hline
Sergue� & \multirow{2}{*}{Impl�mentation basique de l'application serveur} \\
\cline{1-1}
Benoit & \\
\hline
Lionel & \multirow{2}{*}{Impl�mentation basique de l'application client} \\
\cline{1-1}
Nicolas & \\
\hline
\end{tabular}
\end{center}

\subsection{Deuxi�me soutenance}

\begin{center}
\begin{tabular}{||l||l||}
\hline
Sergue� & \multirow{2}{*}{Version op�rationnelle de l'application serveur} \\
\cline{1-1}
Benoit & \\
\hline
Lionel & \multirow{2}{*}{Version op�rationnelle de l'application client} \\
\cline{1-1}
Nicolas & \\
\hline
\end{tabular}
\end{center}

\subsection{Troisi�me soutenance}

\begin{center}
\begin{tabular}{||l||l||}
\hline
Sergue� & \multirow{2}{*}{Bots de gestion et pont NetSoul} \\
\cline{1-1}
Benoit & \\
\hline
Lionel & \multirow{2}{*}{Bots de gestion} \\
\cline{1-1}
Nicolas & \\
\hline
\end{tabular}
\end{center}

\subsection{Soutenance finale}

\begin{center}
\begin{tabular}{||l||l||}
\hline
Sergue� & \multirow{2}{*}{Portage Windows} \\
\cline{1-1}
Benoit & \\
\hline
Lionel & Mode d'emploi \\
\hline
Nicolas & Packaging \\
\hline
\end{tabular}
\end{center}
