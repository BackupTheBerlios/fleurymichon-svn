\section{Nicolas \bsc{Vernot}}

I think The Deer Hunter is the most interesting movie that we had to see because of its approach of the Vietnam war.	Although there is some scene of action, this movie is not based on explosion and gunfight. It try to make the spectator think about what the characters are feeling and why they act in a certain way and not in an other. In particular, the scene where the three characters, that where gone to Vietnam, are captive, allow us to see how people that live in the same environment and shoud so have the same character can react differently to the same situation. The other point in this film that is interesting, is that it show people who seems attached to there eastern-Europe community, but that feel some patriotism, that lead them to go in a war at thousands of kilometers of their house.

\section{Lionel \bsc{Herbin}}

The film which is for me the best of those studied was The Grapes of Wrath. Althought it was a little depressing, this film made me known a little more about the Great Depression in America. I didn't know too much about the Great Depression but thanks to this film I was able to see how it was terrible for some american families. To sum up, a family is forced off of their land and replace them by machines. Then the family has to travel to California to try to find some jobs in order to survive. The Grapes of Wrath shows us how human dignity is hard to keep when a family have to survive against adversity of american society and the law of market. Finally, this film was the one I prefered because John Ford shows us a very touching point of view of the Great Depression.

\section{Benoit \bsc{Menet}}

American Friend of Wim Wenders has appeared to me as the most interesting movie because it is an intelligent mix between action and intrigue. The scenario is quite good, with some key scenes, in order to lead the member of the audience to ask the same questions that Zimmermann's. This man, performed by Bruno Ganz, stays very touching and attaching all along the movie. Indeed, because of his serious blood disease, he becomes Ripley's masterpiece and inevitably falls into a bad and dark story, attracted by the money reward in order to help his family. Ripley, performed by Dennis Hopper, is a subtul and double-faced man who can manipulate people easily, earning their confidence but not losing the control of his fraudulent business practices. However, the movie's production is not good enough to keep the audience awaked, because of these never-ending scenes which considerably slow the movie down.

\section{Sergue� \bsc{Milechine}}

In my opinion, amongst the four movies we had to watch, Wim Wenders' American Friend is the one that needs the longest time to fix the action of the story, because its evolution is very slow, but that doesn't avoid that movie to be the most interesting. To sum up the story, a picture framer suffering from a serious blood disease is manipulated and committed as a murderer in a mafia context, he accepts this job because he thinks that he will die soon so the money is a way for him to help his wife and son after his death. In fact, it is the one I preferred because its plot is very subtle and also because it deals with social aspects of money, such as manipulation, lies and murder. Moreover, the end of movie is quite surprising because the protagonist finally gets to die, that brings out reflections about fate and destiny.
