\section{Introduction}



\section{Concrete examples}

\subsection{Money and sex}



\subsection{Money and social diseases}



\subsection{Money on television}

Within the last decade, various entertainment television shows appeared and became very popular. Some of these shows such as 
\textit{Jackass} or \textit{Dirty Sanchez} work on a very simple concept : the animators are paid to do unusual and stupid 
things, that makes the show funny and attractive especially for teenagers, the age class that is the most interested in 
this kind of programs. However, United States of America do not hold the monopoly in that domain. Indeed, French 
television show animator Cyril Hanouna succeeded in pushing the sin further with his show entitled \textit{La Vie est une 
F�te}. The principle is very simple : he picks two people at random in the street and invites them to participate, saying 
that there is money to win. Then he states the strange challenge they have to take up and announces the starting prize. If 
none of the candidates accepts the challenge, the prize is raised until one of them does accept. Proposed challenges may be 
disgusting, and sometimes funny, for example urinating in his own shoe, eating an alive maggot, putting an alive mouse into 
his pants, asking a policeman to light up a cannabis joint, disturbing an university lesson by walking through the 
classroom with an octopus upon his head or eating a butter brick. Another aspect of money on television is advertising. 
Currently, it is an efficient way for companies to promote the goods and services that they provide, and it is also 
necessary for television channels to maintain their activites. Furthermore, advertisements may influence social trends and 
manage to target a sufficient amount of spectators in order to improve advertiser's income and sales.

\section{Money influence on the settle of identity}

\subsection{About materialism and property}

Money is something important in current social communities. It has been invented to define a standard of value that makes 
trading possible and easy between any amount of commodities. However, money is not omnipotent so materialism is not able to 
solve any problem, for example if someone you rely on gets to die, money and property won't fill the blank left by such a 
disappearance. A study established in Claremont Graduate University, California showed that trust is statistically more 
efficient than competition and individualism. That study confirms Aristotle's thesis saying that man is a social animal. 
Moreover, the young children's idealistic speech that rejects money value system usually does not last very long because 
they get to become aware how non-materialism does draw them aside of their peers. That idea is reinforced by the fact that 
for most citizens who say that they have grown up, adventure and dreams have been replaced by conservatism and conformity.

\subsection{Wealth as power}

Childhood is a period of life when illusions may strongly affect child's future development. A widespread illusion is that 
money is a pure form of power, but it is useful to remind that French expression : <<money doesn't make happiness>>. Reasoning 
this way is false and pledging allegiance to the dollar bill is bad. However, that doesn't mean that money is fully detached 
from power. Indeed, when money is abundant and rightly used, it may prove to be an effective source of power, but it should 
not be considered as power itself. That is well illustrated in Charles Dickens' \textit{Great Expectations} : the main 
character Pip has lost his parents and lives with his sister, who got married with Joe the blacksmith. He encounters with 
an escaped convict who he gets to help. Then he falls in love with Estella, who is a very cold girl and who has been taught 
to break hearts, and wants to impress her, so he needs to become a gentleman. Meanwhile, he gets money for his education from 
an anonymous benefactor, who is in fact Magwitch the convict. After, Pip forgets his family and people that are important to 
him because he becomes corrupted by money, he tries more to impress people than to be moral. Finally, Pip thinks he can fly 
above work and difficulty, so we may notice that his identity has been deeply altered. Nevertheless, Magwitch's plan has not 
failed : he has made Pip become a gentleman and considers self as Pip's second father.

\subsection{The double face of cupidity}

There are two different reasons for one to become greedy. Firstly, one who lives a terrible reality, such as Zimmermann, could be 
drawn into a vicious circle. Indeed, because of his critical situation, he is determined to succeed and is therefore attracted by 
earning a huge reward. So, he becomes excessively obsessed with acquiring money, by any possible means. Secondly, one could easily 
become greedy because of one's education and social class. If one's family needs money, it seems normal that one would be miserly 
and would dream of becoming rich in order to live a better life. In both of these cases, one becomes dependant on money and there 
are risks that one's social relationships would be affected by it.



\section{Conclusion}


