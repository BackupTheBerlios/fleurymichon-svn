\section*{Eiffel 65 - Too Much Of Heaven}

\emph{Song lyrics transcription from album \textbf{Europop}} \\
\begin{small}
Copyright \copyright\ 1999 Universal Music
\end{small}
\\ \\
\noindent
Too much of heaven, \\
Can bring you underground \\
Heaven, can always turn around. \\
Too much of heaven, \\
A life and soul hell bound. \\
Heaven, the killer makes no sound. \\
\\
They're still talking about money, \\
That's right and too much of heaven. \\
C'mon my friend. \\
\\
Let me tell you what it's all about, \\
It's called money dependence today, \\
And people just keep on going on \\
Looking at the dollar bill, \\
And nothing else around them. \\
No love, no friendship, nothing else, \\
Just the dollar bill coming on into their pocket, \\
Into their bank account, \\
And that's too much of heaven \\
Bringing them underground. \\
Let me tell you what it's all about \\
It's called money dependence today, \\
And people just keep on going on \\
Looking at the dollar bill, \\
And nothing else around them. \\
No love, no friendship, nothing else, \\
Just the dollar bill coming on into their pocket, \\
Into their bank account, \\
\\
The answer, \\
Is blowing in the wind. \\
The answer is blowing. \\
\\
Oh let me tell you what it's all about. \\
Too much of heaven bring you underground.

\newpage
\section(Review by Gregory S. Burkart)
\emph(THE AMERICAN FRIEND (1977))
\\
 Author Patricia Highsmith's enterprising sociopath Tom Ripley is prime movie stuff from the sweet smoothness of Rene Clement's 1960 PURPLE NOON, starring ultra-suave Alain Delon, to the more camp (but still fun) approach taken by Anthony Minghella in 1999's THE TALENTED MR. RIPLEY. Highsmith's subsequent novel, RIPLEY'S GAME, is also the subject of two interesting adaptations to date - the latter being Liliana Cavani's recent version with John Malkovich. But for my coin, it's THE AMERICAN FRIEND - recently issued as part of Anchor Bay's Wim Wenders collection - that lingers in the mind as the richest, darkest and classiest cinematic take on this complex character.
\\
This time, none other than Dennis Hopper plays Ripley, with a shaky intensity that reveals more than a little of the wild man Hopper had become at this point in his career. In this version, the manic trickster assumes more of a supporting role. Dramatically, he's eclipsed by the melancholy figure of Jonathan Zimmerman (played by the intense Bruno Ganz of Herzog's NOSFERATU), a likable picture-framer and former art restorer afflicted with a rare and possibly terminal - blood disease. His fascination with vintage toys and novelties leads him to a Hamburg art auction, where he and Ripley have their first uncomfortable encounter. It turns out Ripley's been making a tidy living selling the "lost" paintings of Derwatt (played with acerbic coolness by REBEL WITHOUT A CAUSE director Nicholas Ray), a famous artist whom the world believes dead. And after all, a dead legend outsells a living one.
\\
After being socially snubbed by Zimmerman, and learning of his medical condition, Ripley devises an elaborate revenge, first by spreading the rumor that the man's health has taken a turn for the worse. Zimmerman's growing fear that this may in fact be true eventually drives him to undertake a desperate act in order to insure his family's financial security. He agrees to a proposition made by a French fellow named Raoul (Gerard Blain) to stalk and assassinate a certain underworld figure in Paris - for which he will be paid enough to provide for his wife and kids long after his death. As a further enticement, Raoul offers him the services of a Paris hematologist, which he readily accepts. The lab results are grim, and Jonathan accepts the job out of desperation. Needless to say, he's hardly an expert assassin, but the hit goes down nonetheless.
\\
Hiding the news from his increasingly concerned wife (Lisa Kreuzer), Jonathan comes to realize he's being watched by the unusually friendly Ripley - who seems tormented from a sudden attack of guilt. When Jonathan agrees to participate in another hit at Raoul's behest, Ripley decides to come to his aid. But this job goes completely sour, and leads the two men into a violent face- off with their understandably pissed-off target, a nameless American mobster, played without dialogue by crotchety, cigar-chomping cult director Sam Fuller.
\\
Where RIPLEY'S GAME is basically a traditional thriller, Wenders' adaptation is really only a thriller in the internal sense (a sensibility only a Cold War-era European filmmaker can summon this convincingly). Also, in choosing to put the focus more on Zimmermann than Ripley, the director is essentially creating a spinoff tale, fairly well-removed from the source material, but with a dramatic life of its own.
\\
The soul of the film is definitely Ganz, who turns in a fantastic, heart-wrenching performance. We all know things aren't going to turn out well for Jonathan, and he knows it too. His overwhelming need to provide for his family at any cost shapes him into a flawed but very noble hero, and you can't help but imagine yourself doing the same thing in his position. Wenders wisely puts him at the story's center, creating a dreamlike, strangely depopulated world of constantly-changing languages (the subs sometimes pop on and off three or four times in a single dialogue scene!). In this world, people you don't even know can steer you in painful but inevitable directions.
\\
You can't really fault a character-based film for being a bit sluggish at times, but I'll admit Wenders seldom sidelines the plot for too long. His reasons are usually visual: he wants us to stop and look at little footnotes that enhance Jonathan's life - like a whisper-thin sheet of gold leaf dancing in his hand, the toy cable-cars he keeps colliding with in his son's room, or the little novelty trinkets he and Ripley exchange as a kind of peace offering. These things matter in this world; more so than the ominous industrial equipment that seems to always loom outside every window, or the cold, coffin-like trains filled with numb commuters oblivious to the murder taking place in the next car.
\\
Anchor Bay have revived this art-house fave in fine form. Wenders' color pallette is muted for the most part, but that makes sudden dramatic attacks of color (like the blood-red sunset that follows the first murder) stab at your eyes. I saw very little film grain, considering the �70s stock used, and no color bleeding - which is a relief, since many of the film's most striking images play out as bright objects set against a dark gray ground. The 5.1 mix obviously won't test your system's limits (this is a German art film, remember?) but it does give stirring weight to Jurgen Kneiper's score, which is very Bernard Hermann- flavored.
\\
Nifty extras include a Wenders & Hopper commentary track (too bad Ganz wasn't available - I fuckin' love that guy). It is informative and humorous, but almost too laid-back to keep your attention for two hours. There is also a lengthy reel of cut scenes accompanied by Wenders' commentary, and the theatrical trailer.
\\
I'll admit I'm not really much of a Wenders fan his characters are usually interesting, but engage in way too much navel-gazing for my taste but there's a sad beauty and quiet coolness here that really pulled me in. Sure, it all sounds depressing. Hell, it made me want to jump to my death. But it made me feel something, which most movies these days couldn't' do if they came with free cigars and a blowjob. This is just something that art does. But see it anyway.