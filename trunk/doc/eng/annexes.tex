\chapter{Eiffel 65 - Too Much Of Heaven \cite{chansoneiffel}}

\emph{Song lyrics transcription from album \textbf{Europop}} \\
\begin{small}
Copyright \copyright\ 1999 Universal Music
\end{small}
\\ \\
\noindent
Too much of heaven, \\
Can bring you underground \\
Heaven, can always turn around. \\
Too much of heaven, \\
A life and soul hell bound. \\
Heaven, the killer makes no sound. \\
\\
They're still talking about money, \\
That's right and too much of heaven. \\
C'mon my friend. \\
\\
Let me tell you what it's all about, \\
It's called money dependence today, \\
And people just keep on going on \\
Looking at the dollar bill, \\
And nothing else around them. \\
No love, no friendship, nothing else, \\
Just the dollar bill coming on into their pocket, \\
Into their bank account, \\
And that's too much of heaven \\
Bringing them underground. \\
Let me tell you what it's all about \\
It's called money dependence today, \\
And people just keep on going on \\
Looking at the dollar bill, \\
And nothing else around them. \\
No love, no friendship, nothing else, \\
Just the dollar bill coming on into their pocket, \\
Into their bank account, \\
\\
The answer, \\
Is blowing in the wind. \\
The answer is blowing. \\
\\
Oh let me tell you what it's all about. \\
Too much of heaven bring you underground.

\chapter{Robert Schenk - What is Money? \cite{articleschenk}}

\section{Summary}

Bla

\section{Excerpt}

Economists often use words in ways that are not quite the same as the way the words are used in everyday speech. "Money" is such a word. In everyday speech we use the word "money" in a variety of ways, such as: "My father makes a lot of money," or "John Paul Getty had more money than anyone else," or "General Motors made twice as much money this year as last." In the first sentence above the word money could be replaced with income, in the second with wealth, and in the third with profit or net income. None is the way economists usually use the word "money." The economic definition emphasizes that money is the medium of exchange, or what we use to buy things with. \\ \\
Economic activity can take place without money. All transactions can be barter transactions in which people obtain a good or service that they want by trading away some other good or service that they value less. Because barter is inconvenient, barter systems exist only when exchange is uncommon. Suppose Crusoe has fish and Friday has coconuts. If Crusoe visits Friday to buy coconuts, a trade may not take place. Crusoe wants coconuts, but Friday does not want fish. Exchange will take place only when one of the three realizes that he will have to accept something he does not want but which he can trade later. With only three people and three commodities, this realization will soon take place. However, if there are a hundred people with a hundred different commodities, the pattern of barter transactions necessary for everyone to end up with what he or she wants may be so complex that trade may not occur.1 As trading patterns become more complex, groups need to find a way to reduce the cost of making transactions. They spontaneously begin to use one commodity as an intermediary: they invent money. \\ \\
The invention of money makes trading easier. With money, all prices can be expressed in the same way, in terms of how much money is needed to buy the product. The unit of money becomes the measuring stick of value, or what economists call the standard of value. With a standard of value, computing the costs and benefits of various options, that is, making choices, becomes easier. \\ 
A standard of value is most useful when it does not change over time. If the measuring stick changes with time, comparing costs and benefits of some options may be more difficult. Inflation or deflation change the measuring stick, and a reason people dislike inflation is that it makes comparing options over time more difficult. \\
In addition to its function as a medium of exchange, money also serves as a store of value. Though this function is not what makes money important in macroeconomics, it is vital in explaining how much money people want to hold. Any item that people consider as a way of holding wealth is a store of value. Land, stocks and bonds, old paintings, factories, and jewelry are just some of the other ways people can hold wealth. When money is a good way to hold wealth compared to these alternatives, people will want to hold a lot of it. On the other hand, when money is a poor way to hold wealth, people try to keep little of it. For example, in the German hyperinflation people tried to spend money as soon as they got it because it lost value so quickly. This idea, that people are willing to hold large amounts of money when it is a good store of value, but try to hold small amounts when it is a poor way to hold wealth, is a key idea for those who believe that changes in the amount of money have been an important source of economic disturbance.

