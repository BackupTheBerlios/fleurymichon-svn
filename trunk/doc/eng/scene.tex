\section{The meeting}

\subsection{The approach of Zimmermann}

Firstly the scene takes place in a middle class residential block dark street of Hamburg meanwhile it seemed to happen in the morning because we can see the 
sun shining in the background.
The street is not crowded, there is only a woman with a dog and a white car is parked on the left side of the pavement, so it would be easy for Raoul Minot
to catch Zimmermann when he comes besides him. This man does not walk very fast and he seems to be blas�\footnote{According to Hachette \& Oxford dictonary, 
\textbf{blas�}, \textbf{-e} \textit{adj} blas�.} and not motivated.

\subsection{The trap}

Suddenly, the spectator sees a man coming from a darker perpendicular street where graffitis are tagged on the wall in the corner, he was waiting for 
Zimmermann. He is Raoul Minot and he has been told by Tom Ripley that the man he is going to follow is suffering from a leukemia and may be used to assist 
him for a hit man job. He is a middle aged man and is wearing a brown coat, black pants and shoes and a white scarf. Zimmermann doesn't notice his presence 
and keeps walking toward his art workshop because Minot stays behind.

\subsection{The first contact}

Arrived to the door of Zimmermann's shop, Raoul starts speaking with him with confidence, he appears very sure of himself and tries to win Zimmermann's trust, 
whose look does mainly show uncertainty, surprise and also a little touch of fear. Minot does not know what language to speak with him so he first introduces 
himself in German, the local language, and says his name is Raoul Duplat, which is not his real name. Then he asks him if he speaks French but Zimmermann 
doesn't understand so he continues the conversation in English. After that, he tries to impress him saying that Zimmermann does not know him but he knows 
Zimmermann very well as he says. He convinces him to keep the shop closed and to take the subway in order to talk about a business.


\section{The compulsory proposal in the subway}

\subsection{Minot's false honesty}		

Minot explains very briefly his aim to Zimmermann : he needs to eliminate one or two people, then he says "my cards are on the table" as if he was playing a 
poker game. Zimmermann who is firstly surprised smiles and asks if it is a joke. Seeing that Minot is keeping his seriousness, Zimmermann's smile disappears 
to express a feeling of fear, he becomes very suspicious. He feels menaced so tries to lead the conversation, asking how Minot did find his name.

\subsection{Money as argument}

Minot doesn't even answer to that question and suggests a 250,000 marks reward for the mission he told him about, adding that the job is "safe and easy". 
Zimmermann does not understand why Minot wants to hire him as a gun man and thinks it is a mistake, that he is not the right man. Then Minot has to draw 
more personal arguments in order to convince him.

\subsection{Health as argument}

He seems to be aware of Zimmermann's sickness and says he does not have a long time to live. To shake him harder, he tells him about his family : especially 
he knows that Zimmermann has got a wife and a son, so money is again a way of pressure. Therefore Minot tries to make Zimmermann think that he has to 
accomplish the job for his family. He pretends that Zimmermann's physicist is wrong and that his disease is much serious than he thinks.

\section{The exit}

Exhausted by this heavy conversation, Zimmermann doesn't want to trust Minot's affirmations but that one keeps insisting on the interest of money. Indeed, 
it would allow him to drop his shop and to take profit from his last days on Earth. Finally, Zimmermann leaves the subway but Minot has won his battle : he 
succeeded in spreading doubt within his mind.

