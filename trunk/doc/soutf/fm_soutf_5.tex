\section{BerliOS}

\subsection{Aide au d\'{e}veloppement}

Notre projet est h�berg� chez BerliOS, il s'agit d'un fournisseur de services pour d�veloppeurs qui nous apporte d'innombrables outils pour travailler sur notre projet. Nous disposons d'une interface assez compl�te permettant d'organiser l'�volution du  projet, nous pouvons recevoir des demandes d'ajout de fonctionnalit�s, des demandes de support, �mettre des sondages, rapporter les bugs, discuter, publier les releases du projet, des screenshots, de la documentation \ldots La page d'accueil est accessible � l'adresse : 
\begin{center}
\verb+http://developer.berlios.de/projects/fleurymichon/+
\end{center}

\subsection{Repository}

Par ailleurs, BerliOS nous offre �galement un repository Subversion. Il s'agit d'un gestionnaire de code source, qui nous permet de synchroniser notre travail � la fois rapidement et efficacement. 
\\
TortoiseSVN, un client graphique qui s'int�gre � l'explorateur Windows est disponible gratuitement. Les utilitaires svn et svnadmin sont disponibles avec la majorit� des distributions Linux (\textit{Voir Fig 4.3}). Le WebSVN accessible � l'URL : 
\begin{center}
\verb+http://svn.berlios.de/wsvn/fleurymichon/+ 
\end{center}
permet de consulter les fichiers sources du projet � partir de n'importe quel navigateur.

\section{Site internet}

\subsection{Simple et ergonomique}

Pour un maximum de clarte, le site est programme en HTML et utilise des feuilles de styles CSS pour la mise en forme. Detacher le fond de la forme permet d'optimiser le chargement du site ainsi que son administration. De plus, il respecte la norme W3C et est compatible avec tout les navigateurs. 

\subsection{Base de donn\'{e}es mySQL}

Un systeme dynamique de news (a acces restreint) a ete mis en place afin de communiquer plus facilement avec les utilisateurs de FleuryMichon (eventuelles mise-a-jour, corrections de bugs, \ldots). Celui-ci s'appuie sur une utilisation solide du langage PHP couple a une bas e de donnees mySQL.


 
