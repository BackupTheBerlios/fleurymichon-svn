\section{Vers un serveur plus stable}

Des modifications ont �t� apport�es au serveur afin d'en renforcer la stabilit�. En particulier, l'envoi p�riodique de requ�tes PING aux utilisateurs permet de v�rifier l'�tat de la connexion et par cons�quent d'�liminer les connexions fant�mes. En outre, le traitement des listes d'utilisateurs et de canaux a �t� corrig� car des entr�es n'�taient pas correctement mises � jour lors des actions d'entr�e et de sortie d'utilisateurs sur les canaux.

\section{Vers un client fonctionnel}

L'interface de la version du client de la troisi�me soutenance ne pr�sentait qu'un seul onglet interactif mais d�sormais, les onglets sont g�r�s dynamiquement pour chaque nouveau canal auquel l'utilisateur acc�de. Ainsi, l'application s'adapte au nombre de canaux fr�quent�s par l'utilisateur : lorsqu'il rejoint un canal, un onglet est ajout� � la fen�tre et � l'inverse, lorsqu'il quitte un canal, l'onglet correspondant disparait aussit�t. Par cons�quent, les conversations sont bien s�par�es et les listes d'utilisateurs sont disjointes. Par ailleurs, le client Michon est maintenant capable d'interpr�ter toutes les notifications de base du serveur n�cessaires � l'�tablissement d'�changes simples.
