\documentclass[12pt,a4paper,frenchb]{report}

\usepackage[french]{babel}
\usepackage{a4}
\usepackage[T1]{fontenc}
\usepackage{fancyhdr}
\usepackage{multirow}
\usepackage[official]{eurosym}
\usepackage{graphicx}

%\title{FleuryMichon}
%\author{\emph{Les TouTouYouTou}}

\begin{document}

%\maketitle

\begin{minipage}{\textwidth}
\flushright{
\textbf{\Huge{\bsc{FleuryMichon}}}
\rule[+1.5ex]{\textwidth}{4pt} \\
\emph{\Large{Manuel d'utilisation de Fleury et Michon}}
}
\end{minipage}
\begin{figure}[b]
\begin{center}
\large{
Sergue� \bsc{Milechine} \\
Benoit \bsc{Menet} \\
Lionel \bsc{Herbin} \\
Nicolas \bsc{Vernot} \\
}
\end{center}
\flushleft{
\emph{Les TouTouYouTou}
\rule[+1.5ex]{\textwidth}{2pt}
}
\end{figure}

\newpage

\tableofcontents

\newpage

\chapter{Le serveur Fleury}

\section{installation}

\section{utilisation}


\chapter{Le client Michon}

Michon est un client irc fait selon les normes IRC fixees par la RFC. Ainsi donc il ne devrait poser aucun problemes a communiquer avec tous serveurs irc que 'on pourra trouver faits a cette norme. Les paragraphes suivants vous permettront de vous familiariser avec l'utilisation de ce client.

\section{installation}


\section{utilisation}

Afin d'utiliser irc, l'utilisateur du client doit faire savoir ce qu'il souhaite faire a travers de commandes comprehensibles par le serveur et le client. Les commandes utilisables en question sont enumerees par la suite afin d'indiquer ce qui vous est disponible et comment utiliser le client.

\subsubsection{La commande \emph{connect}}

C'est elle qui va �tablir une connexion avec le serveur (par d�faut \emph{localhost} sur le port 6667), en lui envoyant la combinaison de 
requ�tes NICK, USER et PASS, dont les param�tres seront d�finis par l'utilisateur dans l'onglet \emph{Settings}.
\\\\
\textit{Syntaxe : /connect <adresse de l'h�te> (<port>)}

\subsubsection{La commande \emph{join}}

Lorsque l'utilisateur d�sira cr�er ou rejoindre un canal, il n'aura qu'� utiliser la commande \emph{join}, en lui sp�cifiant en argument le nom du canal 
en question (pr�c�d� d'un di�se).
\\\\
\textit{Syntaxe : /join <canal>}

\subsubsection{La commande \emph{msg}}

La commande \emph{msg} sera utlis�e lorsque l'on d�sirera entrer en communication avec un utilisateur en particulier ou un canal tout entier. 
\\\\
\textit{Syntaxe : /msg <nick ou canal> <texte � envoyer>}

\subsubsection{La commande \emph{part}}

L'utilisateur utilisera la commande \emph{part} lorsqu'il d�sirera quitter ou d�truire un canal.
\\\\
\textit{Syntaxe : /part (<raison>)}

\subsubsection{La commande \emph{kick}}

Si le mod�rateur du canal n'est pas tr�s enjou� par le comportement de certains utlisateurs, il n'aura qu'� effectuer une commande \emph{kick} pour l'�jecter
violemment du dit-canal (� utiliser avec parcimonie sinon cela peut entra�ner une forte d�pendance voir la mort).
\\\\
\textit{Syntaxe : /kick <nick> (<raison>)}

\subsubsection{La commande \emph{quit}}

L'utilisateur se servira de la commande \emph{quit} s'il d�sire quitter le serveur courant.
\\\\
\textit{Syntaxe : /quit}


\end{document}